\PassOptionsToPackage{unicode=true}{hyperref} % options for packages loaded elsewhere
\PassOptionsToPackage{hyphens}{url}
\PassOptionsToPackage{dvipsnames,svgnames*,x11names*}{xcolor}
%
\documentclass[]{article}
\usepackage{lmodern}
\usepackage{amssymb,amsmath}
\usepackage{ifxetex,ifluatex}
\usepackage{fixltx2e} % provides \textsubscript
\ifnum 0\ifxetex 1\fi\ifluatex 1\fi=0 % if pdftex
  \usepackage[T1]{fontenc}
  \usepackage[utf8]{inputenc}
  \usepackage{textcomp} % provides euro and other symbols
\else % if luatex or xelatex
  \usepackage{unicode-math}
  \defaultfontfeatures{Ligatures=TeX,Scale=MatchLowercase}
\fi
% use upquote if available, for straight quotes in verbatim environments
\IfFileExists{upquote.sty}{\usepackage{upquote}}{}
% use microtype if available
\IfFileExists{microtype.sty}{%
\usepackage[]{microtype}
\UseMicrotypeSet[protrusion]{basicmath} % disable protrusion for tt fonts
}{}
\IfFileExists{parskip.sty}{%
\usepackage{parskip}
}{% else
\setlength{\parindent}{0pt}
\setlength{\parskip}{6pt plus 2pt minus 1pt}
}
\usepackage{xcolor}
\usepackage{hyperref}
\hypersetup{
            pdftitle={A Unified Approach for Ex Ante Policy Evaluation},
            pdfauthor={John Graves, Vanderbilt University},
            colorlinks=true,
            linkcolor=Maroon,
            filecolor=Maroon,
            citecolor=Blue,
            urlcolor=blue,
            breaklinks=true}
\urlstyle{same}  % don't use monospace font for urls
\usepackage[margin=1in]{geometry}
\usepackage{graphicx,grffile}
\makeatletter
\def\maxwidth{\ifdim\Gin@nat@width>\linewidth\linewidth\else\Gin@nat@width\fi}
\def\maxheight{\ifdim\Gin@nat@height>\textheight\textheight\else\Gin@nat@height\fi}
\makeatother
% Scale images if necessary, so that they will not overflow the page
% margins by default, and it is still possible to overwrite the defaults
% using explicit options in \includegraphics[width, height, ...]{}
\setkeys{Gin}{width=\maxwidth,height=\maxheight,keepaspectratio}
\setlength{\emergencystretch}{3em}  % prevent overfull lines
\providecommand{\tightlist}{%
  \setlength{\itemsep}{0pt}\setlength{\parskip}{0pt}}
\setcounter{secnumdepth}{5}
% Redefines (sub)paragraphs to behave more like sections
\ifx\paragraph\undefined\else
\let\oldparagraph\paragraph
\renewcommand{\paragraph}[1]{\oldparagraph{#1}\mbox{}}
\fi
\ifx\subparagraph\undefined\else
\let\oldsubparagraph\subparagraph
\renewcommand{\subparagraph}[1]{\oldsubparagraph{#1}\mbox{}}
\fi

% set default figure placement to htbp
\makeatletter
\def\fps@figure{htbp}
\makeatother

\usepackage{setspace}
\doublespacing
\usepackage{etoolbox}
\makeatletter
\providecommand{\subtitle}[1]{% add subtitle to \maketitle
  \apptocmd{\@title}{\par {\large #1 \par}}{}{}
}
\makeatother
\usepackage{booktabs}
\usepackage{longtable}
\usepackage{array}
\usepackage{multirow}
\usepackage{wrapfig}
\usepackage{float}
\usepackage{colortbl}
\usepackage{pdflscape}
\usepackage{tabu}
\usepackage{threeparttable}
\usepackage{threeparttablex}
\usepackage[normalem]{ulem}
\usepackage{makecell}
\usepackage{xcolor}

\title{A Unified Approach for Ex Ante Policy Evaluation}
\providecommand{\subtitle}[1]{}
\subtitle{Supplemental Materials}
\author{John Graves, Vanderbilt University}
\date{}

\begin{document}
\maketitle

\hypertarget{discrete-choice-model}{%
\section{Discrete Choice Model}\label{discrete-choice-model}}

This section describes the structure, assumptions and calibration of a
discrete time and choice model of U.S. health insurance coverage.

To begin, consider a model of insurance choice among \(J\) alternatives
(including the choice not to insure). Define \(U_{itj}\) as the utility
for choice unit \(i\) from selecting choice \(j\) at time \(t\).

\[
U_{itj} = V(\mathbf{x_{itj}}, \mathbf{z_i})+ \epsilon_{ij}
\] \noindent where \(\mathbf{x_{itj}}\) is a vector of time-varying
attributes of the \(J\) choices and the health insurance unit (HIU), or
the collection of related family members who could enroll under the same
plan. Utility also depends on fixed attributes of the HIU
(\(\mathbf{z_i}\)), and an unobservable component \(\epsilon_{itj}\).

For HIU \(i\), the choice of insurance \(y_{it}\) is based on maximizing
utility across the \(J\) alternatives at time \(t\):

\[
y_{it} = {\arg \max}_j [U_{itj}, j = 1, \dots, J]
\]

We next define a function \(B(\cdot)\) mapping utility from choice \(j\)
to \(r_{ij} = P(y_{it} = j)\), the probability of individual \(i\)
selecting choice \(j\). If the error terms \(\epsilon_{ij}\) are
independent across units and are distributed Type I Extreme Value, we
get a standard conditional logit for \(B(\cdot)\). However, other link
functions---such as based on a nested logit or multinomial logit
model---could also be used.

\hypertarget{from-utility-to-probability-linkages-to-common-microsimulation-approaches}{%
\section{From Utility to Probability: Linkages to Common Microsimulation
Approaches}\label{from-utility-to-probability-linkages-to-common-microsimulation-approaches}}

The specification of choice probabilities via a link function to an
underlying utility maximization model is the theoretical chassis for
most major microsimulation models of the U.S. health care system. This
includes models used by the Congressional Budget Office (CB), the RAND
Corporation, and the Urban Institute, among others.

For example, the CBO model utilizes a similar underlying utility
equation:

\[
U_{ij} = \beta_1 V_{ij} + \epsilon_{ij}
\] \noindent where the paramter \(\beta\) rescales utility into dollar
terms. In the CBO model, the systematic component of utility (i.e.,
\(V_{ij}\)) is modeled using microdata on individuals and simulated
employer choices to offer insurance.

\includegraphics[width=0.9\linewidth]{/Users/gravesj/Dropbox/Projects/modeling-health-insurance/./figures/01_cbo-utility-plain}

The CBO model similarly defines a link function \(B(\cdot)\) converting
utility to choice probabilities based a on nested logit in which
individuals first select the \emph{type} of insurance they will have
(e.g., employer, non-group, public, or uninsured) and then conditional
on that choice, select among plans within that type.

In addition to sharing a common ``DNA'' with major microsimulation
models, another nice feature of the discrete choice model structure
outlined above is that it also maps directly into a reduced form
modeling structure. Under this structure, changes in choice
probabilities are modeled using microdata on individuals by applying
reduced form literature-based elasticity estimates to simulated changes
in price. This ``reduced-form'' or ``elasticity-based'' approach to
microsimulation of U.S. health reform was previously used by the CBO
(prior to 2019) and in other major microsimulation models.

\hypertarget{insurance-choice-as-a-markov-process}{%
\section{Insurance Choice as a Markov
Process}\label{insurance-choice-as-a-markov-process}}

As discussed in the introduction to this study, a major downside to
common approaches to microsimulation is that they require considerable h

To address this shortcoming, we next take the utility maximization model
developed above and map it into a ``sufficient statistics'' approach to
modeling changes to U.S. health insurance policy. In so doing we can
summarize policy changes in terms of a minimal set of parameters.

To do so we first recognize that the insurance choice process at two
time periods can be summarized in terms of a Markov trace. First, define
the \emph{ex ante occupancy vector} \(\mathbf{\tilde p}\) summarizing
the count or fraction of the population at baseline.

\begin{itemize}
\item
  Also define a transition probability matrix
  \(\mathbf{R_i} = [r_{irs}]\).

  \begin{itemize}
  \item
    Cells in this \(J \times J\) matrix defined by transition
    probabilities: \(r_{irs} = P(y_{it} = s | y_{i,t-1}=r)\)
  \item
    At a population level (with size \(N\)) define
    \(\mathbf{R} = [r_{rs}]\) where \(r_{rs} = \sum_{i=1}^Nr_{irs}/N\).
  \end{itemize}
\item
  \(\mathbf{p}\), the distribution of coverage at time \(t\), is given
  by \(\mathbf{\tilde{p}'R}\).
\end{itemize}

\hypertarget{development-and-calibration-of-policy-simulation-model}{%
\section{Development and Calibration of Policy Simulation
Model}\label{development-and-calibration-of-policy-simulation-model}}

The basis for the simulation model is longitudianl data on insurance
choice from the 2014 Survey of Income and Program Participation (SIPP)
calibrated to American Community Survey (ACS) on insurance coverage from
2015 to 2018.

\includegraphics[width=0.9\linewidth]{/Users/gravesj/Dropbox/Projects/modeling-health-insurance/figures/99_transprob-results}

\includegraphics[width=0.9\linewidth]{/Users/gravesj/Dropbox/Projects/modeling-health-insurance/figures/99_dd-results}

\begin{verbatim}
## # A tibble: 8 x 6
## # Groups:   non_expansion_state [2]
##   non_expansion_state coverage_type `2015` `2016` `2017` `2018`
##                 <int> <chr>          <dbl>  <dbl>  <dbl>  <dbl>
## 1                   0 01_esi_own    0.633  0.638  0.644  0.650 
## 2                   0 02_priv_oth   0.0854 0.0856 0.0809 0.0772
## 3                   0 03_public     0.180  0.186  0.185  0.182 
## 4                   0 04_uninsured  0.102  0.0896 0.0895 0.0909
## 5                   1 01_esi_own    0.608  0.615  0.620  0.623 
## 6                   1 02_priv_oth   0.102  0.103  0.0936 0.0894
## 7                   1 03_public     0.110  0.112  0.111  0.110 
## 8                   1 04_uninsured  0.180  0.170  0.175  0.177
\end{verbatim}

\begin{verbatim}
## Warning in min(rows_matched): no non-missing arguments to min; returning Inf
\end{verbatim}

\begin{verbatim}
## Warning in max(rows_matched): no non-missing arguments to max; returning -Inf
\end{verbatim}

\includegraphics{01_appendix_ex-ante-evaluation_files/figure-latex/unnamed-chunk-6-1.png}

\begin{verbatim}
## Warning: `cols` is now required.
## Please use `cols = c(p)`

## Warning: `cols` is now required.
## Please use `cols = c(p)`

## Warning: `cols` is now required.
## Please use `cols = c(p)`
\end{verbatim}

\begin{verbatim}
## Warning in min(rows_matched): no non-missing arguments to min; returning Inf
\end{verbatim}

\begin{verbatim}
## Warning in max(rows_matched): no non-missing arguments to max; returning -Inf
\end{verbatim}

\includegraphics{01_appendix_ex-ante-evaluation_files/figure-latex/unnamed-chunk-7-1.png}

\includegraphics{01_appendix_ex-ante-evaluation_files/figure-latex/unnamed-chunk-8-1.png}
\includegraphics{01_appendix_ex-ante-evaluation_files/figure-latex/unnamed-chunk-8-2.png}

\begin{verbatim}
## Warning in bind_rows_(x, .id): binding character and factor vector, coercing
## into character vector

## Warning in bind_rows_(x, .id): binding character and factor vector, coercing
## into character vector

## Warning in bind_rows_(x, .id): binding character and factor vector, coercing
## into character vector
\end{verbatim}

\includegraphics{01_appendix_ex-ante-evaluation_files/figure-latex/unnamed-chunk-8-3.png}

\begin{verbatim}
## Warning: Expected 4 pieces. Missing pieces filled with `NA` in 8 rows [3, 4, 7,
## 8, 11, 12, 13, 14].
\end{verbatim}

\begin{verbatim}
## Warning in min(rows_matched): no non-missing arguments to min; returning Inf
\end{verbatim}

\begin{verbatim}
## Warning in max(rows_matched): no non-missing arguments to max; returning -Inf
\end{verbatim}

\includegraphics{01_appendix_ex-ante-evaluation_files/figure-latex/unnamed-chunk-9-1.png}

\end{document}
