% Options for packages loaded elsewhere
\PassOptionsToPackage{unicode}{hyperref}
\PassOptionsToPackage{hyphens}{url}
\PassOptionsToPackage{dvipsnames,svgnames*,x11names*}{xcolor}
%
\documentclass[
  10pt,
]{article}
\usepackage{lmodern}
\usepackage{amssymb,amsmath}
\usepackage{ifxetex,ifluatex}
\ifnum 0\ifxetex 1\fi\ifluatex 1\fi=0 % if pdftex
  \usepackage[T1]{fontenc}
  \usepackage[utf8]{inputenc}
  \usepackage{textcomp} % provide euro and other symbols
\else % if luatex or xetex
  \usepackage{unicode-math}
  \defaultfontfeatures{Scale=MatchLowercase}
  \defaultfontfeatures[\rmfamily]{Ligatures=TeX,Scale=1}
\fi
% Use upquote if available, for straight quotes in verbatim environments
\IfFileExists{upquote.sty}{\usepackage{upquote}}{}
\IfFileExists{microtype.sty}{% use microtype if available
  \usepackage[]{microtype}
  \UseMicrotypeSet[protrusion]{basicmath} % disable protrusion for tt fonts
}{}
\makeatletter
\@ifundefined{KOMAClassName}{% if non-KOMA class
  \IfFileExists{parskip.sty}{%
    \usepackage{parskip}
  }{% else
    \setlength{\parindent}{0pt}
    \setlength{\parskip}{6pt plus 2pt minus 1pt}}
}{% if KOMA class
  \KOMAoptions{parskip=half}}
\makeatother
\usepackage{xcolor}
\IfFileExists{xurl.sty}{\usepackage{xurl}}{} % add URL line breaks if available
\IfFileExists{bookmark.sty}{\usepackage{bookmark}}{\usepackage{hyperref}}
\hypersetup{
  pdftitle={Supplemental Appendix: A Unified Approach for Ex Ante Policy Evaluation},
  pdfauthor={John Graves, Vanderbilt University},
  colorlinks=true,
  linkcolor=Maroon,
  filecolor=Maroon,
  citecolor=Blue,
  urlcolor=blue,
  pdfcreator={LaTeX via pandoc}}
\urlstyle{same} % disable monospaced font for URLs
\usepackage[margin=1in]{geometry}
\usepackage{graphicx,grffile}
\makeatletter
\def\maxwidth{\ifdim\Gin@nat@width>\linewidth\linewidth\else\Gin@nat@width\fi}
\def\maxheight{\ifdim\Gin@nat@height>\textheight\textheight\else\Gin@nat@height\fi}
\makeatother
% Scale images if necessary, so that they will not overflow the page
% margins by default, and it is still possible to overwrite the defaults
% using explicit options in \includegraphics[width, height, ...]{}
\setkeys{Gin}{width=\maxwidth,height=\maxheight,keepaspectratio}
% Set default figure placement to htbp
\makeatletter
\def\fps@figure{htbp}
\makeatother
\setlength{\emergencystretch}{3em} % prevent overfull lines
\providecommand{\tightlist}{%
  \setlength{\itemsep}{0pt}\setlength{\parskip}{0pt}}
\setcounter{secnumdepth}{5}
\usepackage{setspace}
\doublespacing
\usepackage{amsmath}
\usepackage{booktabs}
\usepackage{longtable}
\usepackage{array}
\usepackage{multirow}
\usepackage{wrapfig}
\usepackage{float}
\usepackage{colortbl}
\usepackage{pdflscape}
\usepackage{tabu}
\usepackage{threeparttable}
\usepackage{threeparttablex}
\usepackage[normalem]{ulem}
\usepackage{makecell}
\usepackage{xcolor}

\title{Supplemental Appendix: A Unified Approach for Ex Ante Policy Evaluation}
\usepackage{etoolbox}
\makeatletter
\providecommand{\subtitle}[1]{% add subtitle to \maketitle
  \apptocmd{\@title}{\par {\large #1 \par}}{}{}
}
\makeatother
\subtitle{Evaluating Mechanisms for Universal Health Coverage}
\author{John Graves, Vanderbilt University}
\date{}

\begin{document}
\maketitle

\centering
\raggedright

\hypertarget{modeling-coverage-changes}{%
\section{Modeling Coverage Changes}\label{modeling-coverage-changes}}

Recall from the main text that the impact of a modeled reform on
coverage is summarized as

\begin{equation}
\label{eq:takeup_potout}
  \boldsymbol{\theta} = \boldsymbol{p(1)} -  \boldsymbol{p(0)} 
  = \boldsymbol{\tilde{p}'R(1)} - \boldsymbol{\tilde{p}'R(0)}
\end{equation}

\noindent where \(\boldsymbol{\tilde p}\) is the \emph{ex ante occupancy
vector} summarizing the count or fraction of the population in each
coverage category at time \(t-1\) (i.e., at baseline). The transition
probability matrix is defined as \(\boldsymbol{R_i} = [r_{irs}]\). Cells
in this \(J \times J\) matrix are defined by transition probabilities
among the \(J\) possible coverage categories based on conditional choice
probabilities: \(r_{irs} = P(y_{it} = s | y_{i,t-1}=r)\).

\hypertarget{estimating-boldsymboltilde-p}{%
\subsection{\texorpdfstring{Estimating
\(\boldsymbol{\tilde p}\)}{Estimating \textbackslash boldsymbol\{\textbackslash tilde p\}}}\label{estimating-boldsymboltilde-p}}

The ex ante occupany vector simply summarizes the fraction or count of
the target population in each coverage category at baseline. We can
therefore appeal to survey samples to obtain estimates. In our results
we estimate \(\boldsymbol{\tilde p}\) based on the 2014 panel of the
Survey of Income and Program Participation (SIPP). The SIPP provides us
with a baseline distribution of coverage in 2015---one year after the
Affordable Care Act's (ACA) major coverage reforms went into place.

\hypertarget{estimating-and-calibrating-boldsymbolr0}{%
\subsection{\texorpdfstring{Estimating and Calibrating
\(\boldsymbol{R(0)}\)}{Estimating and Calibrating \textbackslash boldsymbol\{R(0)\}}}\label{estimating-and-calibrating-boldsymbolr0}}

We begin by first specifying a process for estimating and then
calibrating the baseline transition probability matrix
\(\boldsymbol{R(0)}\) to match observed population totals on the
evolution distribution of coverage over time.

\begin{itemize}
\tightlist
\item
  Basis is 2014 SIPP
\item
  Use SIPPP to non-parameterically estimate transiton hazards among
  coverage types. Basis for this is a multi-state model as outlined in
  Graves and Nikpay (2017)
\end{itemize}

\[
\boldsymbol{R}(0) = 
\begin{pmatrix}
0.881 & 0.005 & 0.05 & 0.065 \\
0.151 & 0.665 & 0.126 & 0.058 \\
0.145 & 0.059 & 0.503 & 0.293 \\
0.294 & 0.163 & 0.295 & 0.249 \\
\end{pmatrix}
\]

\begin{figure}
\includegraphics[width=1\linewidth]{/Users/gravesj/Dropbox/Projects/modeling-health-insurance/./figures/01-posterior-distribution} \caption{\label{posterior_calib}Posterior Distribution of Calibrated Transition Probabilities}\label{fig:unnamed-chunk-4}
\end{figure}

\begin{figure}
\includegraphics[width=1\linewidth]{/Users/gravesj/Dropbox/Projects/modeling-health-insurance/./figures/01-calibration-base-model} \caption{\label{posterior_calib}Calibration Plot for Distribution of Insurance Coverage by Year}\label{fig:unnamed-chunk-5}
\end{figure}

\hypertarget{modeling-boldsymbolr1-link-to-cbo-and-other-microsimulation-models}{%
\subsection{\texorpdfstring{Modeling \(\boldsymbol{R(1)}\): Link to CBO
and Other Microsimulation
Models}{Modeling \textbackslash boldsymbol\{R(1)\}: Link to CBO and Other Microsimulation Models}}\label{modeling-boldsymbolr1-link-to-cbo-and-other-microsimulation-models}}

Estimation or modeling of the transition probabilities can be
accomplished several ways: by estimating or deriving them via
literature-based reduced form evidence, or by modeling them directly
using microsimulation. While our application relies on the reduced-form
approach, we will first discuss briefly how this can be accomplished in
a microsimulation model here.

A standard assumption is that an exogenous policy change does not affect
the unobserved disturbance term \(\epsilon_{itj}\) in the underlying
discrete choice formulation:

\begin{equation}
\label{eq:utility_1}
U_{itj} = V(\mathbf{x_{itj}}, \mathbf{z_i})+ \epsilon_{itj}
\end{equation}

\noindent where \(\mathbf{x_{itj}}\) is a vector of time-varying
attributes of the \(J\) choices and the health insurance unit (HIU), or
the collection of related family members who could enroll under the same
plan. Utility also depends on fixed attributes of the HIU
(\(\mathbf{z_i}\)), and an unobservable component \(\epsilon_{itj}\). A
function \(B(\cdot)\) maps utility from choice \(j\) to
\(r_{ij} = P(y_{it} = j)\), the probability of individual \(i\)
selecting choice \(j\).

The specification of choice probabilities via a link function to an
underlying utility maximization model is the theoretical chassis for
most major microsimulation models of the U.S. health care system. This
includes models used by the Congressional Budget Office (CB), the RAND
Corporation, and the Urban Institute, among others.

For example, the CBO model utilizes a similar underlying utility
equation:

\begin{equation}
\label{eq:utility_cbo}
U_{ij} = \beta_1 V_{ij} + \epsilon_{ij}
\end{equation}

\noindent where the parameter \(\beta\) rescales utility into dollar
terms.

The CBO health reform model similarly defines a link function
\(B(\cdot)\) converting utility to choice probabilities based a on
nested logit in which individuals first select the \emph{type} of
insurance they will have (e.g., employer, non-group, public, or
uninsured) and then conditional on that choice, select among plans
within that type. One difference, however, is that the CBO only models
marginal changes in coverage, not transitions as we do here. In that
sense the CBO model yields estimates analogous to repeated cross section
data, versus longitudinal data.

With the underlying utility structure specified, policy changes are then
modeled to affect utility/take-up through their impact on prices,
quality, offers of employment-based insurance, etc. In a microsimulation
model, this affects the systematic component of utility (i.e.,
\(V_{ij}\)), which is modeled directly using calibrated microdata on
individuals and simulated employer choices to offer insurance.
Attributes of plans and individuals in the microdata are adjusted to
reflect the modeled reform scenario. Specific parameters in the
systematic component of the CBO microsimulation model are summarized in
the equation below.

\includegraphics[width=0.7\linewidth]{/Users/gravesj/Dropbox/Projects/modeling-health-insurance/./figures/01_cbo-utility-plain}

Similarly, in an elasticity-based microsimulation model---which the CBO
used prior to 2018---price changes for each of the \(J\) insurance
options are simulated for units in the microdata. Elasticities and
further adjustments (e.g., income effects) are then applied to derive
new choice probabilities. These aggregated choice probabilities, along
with attributes of individuals (e.g., health status) and policy (e.g.,
subsidy schedules) are the building blocks for other modeled outcome
changes (e.g., cost of subsidies, premiums, etc.). For example, the
Gruber Microsimulation Model, which was used by the White House and
Congress to model the ACA, used an underlying reduced-form take-up
equation with the following form: \[
P(y_{it} = j) = (\texttt{Constant} + \texttt{Elasticity} \times \texttt{Percent Price Change} \times \texttt{Income Effect})*\texttt{Income Adjustment}
\]

\hypertarget{modeling-boldsymbolr1-a-reduced-form-approach}{%
\subsection{\texorpdfstring{Modeling \(\boldsymbol{R(1)}\): A
Reduced-Form
Approach}{Modeling \textbackslash boldsymbol\{R(1)\}: A Reduced-Form Approach}}\label{modeling-boldsymbolr1-a-reduced-form-approach}}

A nice feature of the modeling structure developed here is that
researchers can simply estimate or derive takeup probabilities from the
applied literature rather than use a detailed microsimulation model. In
this section we will detail how we derive estimates of
\(\boldsymbol{R_i}\) using differences-in-differences (for estimates of
the impact of public program expansion) and regression discontinuity
(for estimates of take-up of subsidized private plans).

\hypertarget{refs}{}
\leavevmode\hypertarget{ref-gravesChangingDynamicsUS2017}{}%
Graves, John A., and Sayeh S. Nikpay. 2017. ``The Changing Dynamics of
US Health Insurance and Implications for the Future of the Affordable
Care Act.'' \emph{Health Affairs} 36 (2): 297--305.
\url{https://doi.org/10.1377/hlthaff.2016.1165}.

\end{document}
